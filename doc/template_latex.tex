\documentclass[10pt,a4paper,twocolumn]{article}

%\usepackage{epcg_en}
% Versao em Portugu�es 
\usepackage{epcg_pt}

\usepackage{times}
\usepackage{epsfig}
\usepackage[latin1]{inputenc}
\pagestyle{empty}

\title{Portuguese Annual Computer Graphics Meetings \\
Submission Format}

\author{Nome1 Apelido1 \\
Dep. XYZ, Institui��o \\
Av. Qualquer, 1000 Lisboa \\
\small{\texttt{email@dominio.pt}}\\
\and
Nome2 A. Apelido2 \\
Univ. XYZ \\
Cidade  \\
\small{\texttt{email2@dominio.pt}}\\ 
\and
Nome3 Apelido3 \\
Institui��o X / Institui��o Y \\
Cidade  \\
\small{\texttt{email3@dominio.pt}}\\
\and
Nome4 Apelido4 ~~~~~ Nome5 Apelido5 \\
Uma Mesma Institui��o \\
Rua Qualquer, 4000 Porto \\
\small{\texttt{\{email4,email5\}@dominio.pt}}}

\date{}

\abstract{Em artigos escritos noutra l�ngua que n�o a portuguesa, deve ser inclu�do um resumo em portugu�s / A portuguese abstract should be present in non-portuguese submissions.
}

% if no additional abstract is needed, comment or remove the following lines
\additionalabstracttitle{Abstract}
\additionalabstract{
This document describes the format of submissions to Portuguese Annual Computer Graphics Meetings.
The abstract should be written using 10-point Times New Roman (hereinafter
designated as Times) or equivalent (e.g. Computer Modern) with justified right
margin and both margins indented by 0,5cm with respect to body text. The
Abstract should be delimited by two horizontal lines 1 1/2 point wide, set 12
points off text. The abstract should be separated from text above and below by
24 points (at least).
}


\keywords{A list of up to seven keywords, separated by commas. 10-point Times
New Roman justified.}

\begin{document}
\maketitle
\thispagestyle{empty}

\Section{INTRODUCTION}
This document describes a format for contributions sub-mitted to the Portuguese Annual Computer Graphics Meetings (EPCG). 

Contributions should be submitted in electronic format (preferably PDF). Other formats cannot be accepted unless previously agreed with the editors.


\Section{MULTIMEDIA ATTACHMENTS}
We encourage authors to submit multimedia attachments using the more common
formats (JPEG, MPEG, VRML, etc.) to complement and enrich the textual
submissions. For appendices larger than 10Mb, you should ask for the editor's
agreement beforehand.

\Section{BODY TEXT and MARGINS}
Documents should be formatted to A4 page size, with a  1,9 cm left and right
margins. Top and bottom margins should be 2,5 cm. Header and footnotes should be
set off 1,25 cm from margins. \textbf{Do not number} pages. Contributions to
EPCG should \textbf{not} use headers or footers.

Body text should be formatted as two columns 8,18 cm wide, spaced 0,84 cm apart.
In the last page, both columns should have the same height, as show here.

\Section{TEXT FORMATTING GUIDELINES}
Text format should follow the guidelines set forth in the following sections wrt
organisation, layout and style.  Table 1, shown in appendix summarises the
styles used in this document.

\SubSection{Normal Text}
Normal text should be justified on both margins using 10-point \textit{Times}
font or equivalent and single-space lines.

Space between paragraphs should be 4 points.

A numbered list is made of normal text, where a number sets off each line:

\begin{enumerate} \item Items in numbered lists should be indented by 0,63 cm.
\end{enumerate}

Bulleted lists also use normal text, where each line is preceded by a special
symbol:
\begin{itemize}
\item Text in a bulleted list is also indented by 0,63 cm.
\end{itemize}

Only in special cases should be used a different font, such as when presenting
program source code, for which a san-serif fixed-width font such as 10-point
\texttt{Courier} or \texttt{Courier New} or equivalent is recommended.

\SubSection{Title and Author Information}
After the paper Title (Times 18 point bold), there should appear author names
(Times 12 point font), affiliations (10pt Helvetica) and e-mail addresses is any
(Courier 10 pt) formatted to a maximum of three equi-spaced columns of the same
width, at the top of the document's first page. If there are more than three
co-authors, the layout should use multiple lines as shown in this example.
Common affiliations should be centred across the page as illustrated by the last
two authors in this example text.

\SubSection{Citations and References}
Footnotes\footnote{Footnotes should appear below the column containing the
reference.} should be set using 9-point Times left and right justified.

All bibliographic references should use the same style and appear alphabetically
ordered by first author's surname (example \cite{mantyla88, meagher82}). References should use the \texttt{long\_en} format. 

\Section{SECTIONS}
Section headings should be set in Helvetica 10-point bold font in ALL CAPS, left
justified. Sections at the end of a column with less 4 lines of text, should
start in the next column.

\SubSection{Subsection Headings}
Subsection headings should be set in Helvetica 10 point bold with all words
Capitalised (Note: particles such as  "a" and "the" should not be capitalised
unless they are the first word.)
\SubSubSection{Subsubsection headings}
Use Helvetica italic 10-point font. Only the first word should be capitalised.

\Section{FIGURES, TABLES, IMAGES and CAPTIONS}
Figures, tables and images should be placed as close as possible to the text
that references them. These elements should not appear without being referred to
in the body text or without a caption. Whenever possible, these should be as
wide as a single column or appear centred within a column if narrower.
\textit{If necessary}, these elements can take the full two-column width,
\textit{without exceeding the document margins}.

Captions should use Times 9 point boldface, and should be numbered (e.g: "Table
1", "Figure 1") and centred immediately below their corresponding image, table
or figure.

\Section{ACKNOWLEDGEMENTS}
If there are acknowledgements, these should appear in the last section,
immediately before the references.


\bibliography{bibliog}
%\bibliographystyle{long_pt}
\bibliographystyle{long_en}

\end{document}