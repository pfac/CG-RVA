\Section{INTRODUCTION}
\label{sec:intro}

ARToolKit is a library implemented in C/C++ for building Augmented Reality (AR) applications. Due to the abstraction and mechanisms it provides, several ports of this library have been made to other languages, extending its functionalities to other programmers. The nyARToolKit was a direct port of the original library to the Java language. From this, the FLARToolKit library was born, which ported the functionalities to ActionScript (Flash). This last port was then used by Ilmari Heikkinen to create the library behind this project: the JSARToolKit, a port of the FLARToolKit library to build AR applications in Javascript, allowing this kind of applications to be deployed in a more portable way, via Web Browsers, and possibly smartphones.

This document describes the implementation of an interactive application which follows a mechanic analogous to that of the role-playing game Pokémon\textsuperscript{\textregistered}, using the JSARToolKit library to implement the interactive component using Augmented Reality. This application revolves the original game's premise where the goal is to collect all the different Pokémons, but the concept is simplified: the goal is to capture the monsters which show up randomly in the markers by simply placing a hand on the marker (obstructing it's visibility). The game mechanics is meant for continuous playing around a scoring system limited by time.

\cref{sec:jsartoolkit} describes how the JSARToolKit library was used to identify the markers in the player webcam stream and how WebGL was used to add virtual elements to the scene. In \cref{sec:game}, the game mechanics, how the markers are managed and the scoring system are explained. The conclusion is presented in \cref{sec:conclusion}. Additionally, a description of the setup required to test the game is presented in \cref{sec:setup}.
